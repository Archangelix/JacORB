
In this chapter  we explain how to obtain and  install JacORB, and give
an overview of the package contents.

\section{Downloading JacORB}

JacORB can be downloaded as a source or binary distribution in a g-zipped
tar--archive or zip--archive format from the JacORB home page at
\href{http://www.jacorb.org}{http://www.jacorb.org}. It is also available from
Maven Central (See \ref{ch:mvn}). To install JacORB, first unzip and untar (or
simply unzip) the archive somewhere.  This will result in a new directory {\tt
\JacORBDir}.

\section{Installation}
\label{Sec_installation}

\subsection{Requirements}

JacORB requires JDK 1.6 or above properly installed on your machine.  To build
JacORB (and compile the examples) you need to have the XML--based make tool
``Ant'' installed on your machine (version 1.7.1 or greater).  Ant can be downloaded
from \href{http://ant.apache.org}{http://ant.apache.org}. All make
files ({\tt build.xml}) are written for this tool.

\subsection{Procedure}
\begin{itemize}
\item No further steps are required if using the binary download of JacORB.
\item To configure a jacorb.properties file (using the etc/jacorb\_properties.template) see \ref{ch:configuration}.
\item If using a source download, to rebuild JacORB completely see~\ref{sec:building}.
\end{itemize}

For SSL, you need an implementation of the SSL protocol. We currently support
Oracle's JSSE Reference implementation included in the JDK.

\subsection{Libraries}
Once JacORB is installed, the bin directory should contain the scripts used to assist in
running JacORB applications (see \ref{runningExampleApp} and \ref{appRunningEndorsed}).
The lib directory contains the JacORB libraries and third party libraries required by the ORB
and services (See below). JacORB libraries have been split as follows:

\begin{itemize}
\item jacorb.jar          - containing the ORB, IMR, IR and NamingService
\item jacorb-omgapi.jar   - containing the core OMG API stubs.
\item jacorb-services.jar - containing all other services (e.g. Notification, DDS, Collection etc).
\item idl.jar             - containing the IDL compiler.
\end{itemize}

\subsection{Dependencies}

JacORB depends upon the following third party software
\begin{itemize}
\item Simple Logging Facade For Java (SLF4J Version 1.6.4)
\end{itemize}

Note that the services may depend upon further third party libraries.

\subsection{Building JacORB}
\label{sec:building}

Ensure that Ant is installed on your system and that you are starting with a
clean environment. The easiest way to start is with your CLASSPATH
unset. Windows users should ensure that JacORB is installed in a path that does
not contain spaces. Errors will be encountered building the Notification Service
otherwise.

To build JacORB type

\cmdline{ant all}

in the installation directory. Optionally, you might want to first run

\cmdline {ant realclean}.

By default debugging is turned on when building JacORB and set to {\tt lines,source,vars}
\paragraph{Notes:}
The build process may abort, claiming that javac does not have enough
memory. This can happen when trying to compile a large number of files at one
time. Try setting the variable

\cmdline{ANT\_OPTS=-Xmx500M}

\subsubsection{Building JacORB Demos (Optional)}
To build the JacORB demos within the demo directory a build.xml file
has been supplied. Similar to above, simply call "ant" in the demo directory.
The default Ant target will build all the demos. The classes will be placed
in {\tt <installdir>/classes}

\subsubsection{Building JBoss Service for the Notification Service (Optional)}
In order to build the JBoss service (jboss-cosnotification.sar in the lib
directory), the paths to JBOSS\_HOME for the
\href{http://www.jboss.org/jbossas}{JBoss Application Server} and MX4J\_HOME
for the \href{http://mx4j.sourceforge.net}{MX4J implementation}
 of the Java Management Extensions (JMX) have to be specified for the ant build
process. There are two possible alternatives for this task.
\begin{itemize}
\item Set the paths in the environment variables named JBOSS\_HOME and MX4J\_HOME respectively.
\item Call ant with the additional parameters {\tt -Djboss.home=<path-to-jboss-installation> -Dmx4j.home=<path-to-mx4j-installation>}
\end{itemize}


\section{Versioning}
There are a variety of ways of determining the version of JacORB that is being used.
\begin{itemize}
\item The configuration option {\tt jacorb.orb.print\_version} may be used to output the version to the logfile when JacORB is running
\item The IDL compiler '-v' will output the version of the compiler/JacORB.
\item The command {\tt jaco org.jacorb.util.Version} may be run to output the current version and build date.
\end{itemize}

%%% Local Variables:
%%% mode: latex
%%% TeX-master: "../ProgrammingGuide"
%%% End:
