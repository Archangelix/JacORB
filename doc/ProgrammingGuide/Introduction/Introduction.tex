%
% $Id$
%

This document gives an introduction to programming distributed
applications with JacORB, a free Java object request broker. JacORB
comes with full source code, a couple of CORBA Object Service
implementations, and a number of example programs.  The JacORB version
described in this document is JacORB \JacORBVersion.

\section{A Brief CORBA introduction}

The idea behind CORBA is to model distributed resources as objects
that provide a well-defined interface, and to invoke services through
remote invocations (RPCs). Since the transfer syntax for
sending messages to objects is strictly defined, it is possible to
exchange requests and replies between processes running program
written in arbitrary programming languages and hosted on arbitary
hardware and operating systems. Target addresses are represented as
{\em Interoperable Object References} (IORs), which contain transport
addresses as well as identifiers needed to dispatch incoming messages
to implementations. 

Interfaces to remote objects are described declaratively in an
programming language-independent {\em Interface Definition Language}
(IDL), which can be used to automatically generate language-specific
stub code.

It is important to stress that:
\begin{itemize}
\item CORBA objects are abstract entities  seen by clients and
  represented by artifacts in potentially arbitrary, even non-OO
  languages. These artifacts are calles {\em servants} in CORBA
  terminology.
\item CORBA objects achieve location transparency, i.e., clients need
  not be (and generally are not) aware of the actual target hosts
  where servants reside. However, complete distribution transparancy
  is not achieved in the sense that clients would not notice a
  difference between a local function call and a remote CORBA
  invocation. This is due to factors such as increased latency,
  network error conditions, and CORBA-specific initialization code in
  applications, and data type mappings.
\end{itemize}

Please see \cite{Brose2001a,Siegel2000, Vinoski1997} for more
information and additiomal details, and \cite{Henning1999} for
advanced issues.

\section{Project History}

JacORB originated in 1995 (was it 1996?) in the CS department at Freie
Universit{\"a} Berlin (FUB). It evolved from a small Java RPC library and a
stub compiler that would process Java interfaces. This predecessor was
written --- most for fun and out of curiosity --- by Boris Bokowski
and Gerald Brose because at that time no Java RMI was available. The
tow of us then realized how close the Java interface syntax was to
CORBA IDL, so we wrote an IDL grammar for our parser generator and
moved to GIOP and IIOP as the transport protocol. It was shortly
before Christmas 1996 when the first interoperable GIOP request was
sent form a JacORB client to an IONA Orbix server. For a long time,
JacORB was the only free (in the GNU sense) Java/CORBA implementation
avialable, and it soon enjoyed widespread interest, at first mostly in
academic projects, but commercial use followed soon after.

For a while, Gerald developed JacORB as a one-man-project until a few
student projects and master theses started adding to it, most notably
Reimo Tiedemann's POA implementation, and Nicolas Noffke's
Implementation Repository and Portable Interceptor implementations.
Other early contributors were Sebastian M�ller, who wrote the
Appligator, and Herbert Kiefer, who added a policy domain service
(which is no longer part of the JacORB distribution). 

A more recent addition is Alphonse Bendt's implementation of the CORBA
Notification Services as part of his master's theses. Substantial
additions to the JacORB core were made by Andr� Spiegel, who
contributed OBV and AMI implementations. Other substantial
contributions to JacORB have been added over time by the team at
PrismTech UK (Steve Osselton, Nick Cross, Simon McQueen, Jason
Courage).

JacORB continues to be used for research at FUB, especially in the
field of distributed object security. Even though a number of people
from the core team have left FUB (Gerald, Nico, and Reimo are now with
Xtradyne Technologies, Andr� Spiegel is now a freelance developer and
consultant), the JacORB project is still rooted at Freie
Universit{\"a}t Berlin, which hosts the JacORB web and CVS server.

Due to the limited number of developers, the philosophy around the
development has never been to achieve feature-completeness beyond the
core 90\%, but standards compliance and quality.  (e.g., JacORB 2.0
does not come with a PolicyManager).  Brand-new and less widely-used
features had to wait until the specification had reached a minimum
maturity --- or until somene offered project funding.

\section{Support}

The JacORB core team and the user community together provide best
effort support over our mailing lists. 

To enquire about commercial support, please send email to {\tt
  info@jacorb.com} if you want members of the JacORB core team.
Commercial support is also available from PrismTech and OCI.

\section{Contributing --- Donations}

In essence, the early development years were entirely funded by public
research. JacORB did receive some sponsoring over the years, but not
as much as would have been desirable. A few development tasks that
would otherwise not have been possible could be payed for, but more
would have been possible --- and still is. 

If you feel that returning some of the value created by the use of
Open Source software in your company is a wise investment in the
future of that the software (maintenance, quality improvements,
further development) in the future, then you should contact us about
donations.

Buying hardware and sending it to us is one option. It is also
possible to directly donate money to the JacORB project at Freie
Universit{\"a}t Berlin. If approval for outright donations is
difficult to obtain at your company, we can send you an invoice for,
e.g.., CORBA consulting.

\section{Contributing --- Development}

If you want to contribute to the development of the software directly,
you should do the following:

\begin{itemize}
\item download JacORB and run the software to gain some first-hand
  expertise first 
\item read this document and other sources of CORBA documentation,
  such as \cite{Brose2001a}, and the OMG's set of specifications
  (CORBA spec., IDL/Java language mapping)
\item start reading the code
\item subscribe to the {\tt jacorb-developer} mailing list to share
  your expertise
\item contact us to get subscribed to the core team's mailing list and
  gain CVS access
\item read the coding guide line
\item contribute code and test cases
\end{itemize}

%%% Local Variables: 
%%% mode: latex
%%% TeX-master: "../ProgrammingGuide"
%%% End: 

